\documentclass{../cpct/ctsol}

\title{ACM 算法实验室 2022 年 12 月月赛题解}
\date{2022 年 12 月 4 日}

\begin{document}

\maketitle
\addsolution{Quento}{AgOH}{dfs}
\addsolution{地铁 \textasciitilde 地铁 \textasciitilde}{Algor}{图的连通性}
\addsolution{数列 \textasciitilde 数列 \textasciitilde}{Algor}{签到}
\addsolution{布丁 \textasciitilde 布丁 \textasciitilde}{Algor}{贪心}
\addsolution{排队 \textasciitilde 排队 \textasciitilde}{Algor}{逆序对}
\addsolution{背包 \textasciitilde 背包 \textasciitilde}{Algor}{构造}
\addsolution{Guesslang}{Zxilly}{模拟}
\addsolution{聚会 \textasciitilde 聚会 \textasciitilde}{Algor}{二分答案}

\section*{题目概览}

\solutiontab

\makesolution
\section*{做法}

对于每个“Quento”棋盘格,从每个数字格为起点做 dfs,统计所有可能出现的结果,对于每个询问查询结果即可。

\section*{标程}

\std{A}

\makesolution
\section*{做法}

无向非连通图上判断给定两点是否连通。

1.并查集做法:运用并查集在读入数据的同时 $merge(u_1,v_i)$ 最后只需判断 $find(s) == find(t)$ 即可。

2.dfs做法:链式前向星或者邻接表存图,从给定两点中的任意一点开始dfs,同时记录 $vis$ 数组,最后只要判断另一点的 $vis$ 值是否为 $true$ 即可。

\section*{标程}

\std{B}

\makesolution
\section*{做法}

签到题,如题意模拟即可。

\section*{标程}

\std{C}

\makesolution
\section*{做法}

贪心加模拟,尽量早吃甜度较大的布丁。

\section*{标程}

\std{D}

\makesolution
\section*{做法}

给你一个数列求逆序对的数量。当 $i < j \cap a_i > a_j$ 时 $\{i,j\}$ 称为一个逆序对。

归并排序模板简单改动,仅需在 $merge()$ 的时候判断当前是否产生逆序对。

\section*{标程}

\std{E}

\makesolution
\section*{做法}

给出一个数列,要求你更改其顺序使得该数列没有任何一个前缀的和等于 $m$。

只有当数列的总和等于 $m$ ,或者总和大于 $m$ 并且数列中每个数都相等且为 $m$ 的因子时无解。

即当且仅当以下式子成立时,该数列无解:

$$
\sum_{i=1}^{n}a_i=m \cup  \sum_{i=1}^{n}a_i>m  \cap \forall a_i~(1 \leq i < n),a_i=a_{i+1} \cap a_1 | m
$$

其他情况均有解,只需要依次累加 $a_i$,当 $sum=m$ 的时候用不同于 $a_i$ 的其他数代替即可。 

\section*{标程}

\std{F}

\makesolution
\section*{做法}

统计每个可能出现的关键字的可能出现的语言集合,例如若 \texttt{C++} 语言中有关键字 \texttt{if} 和 \texttt{else},\texttt{Java} 语言中有关键字 \texttt{if},\texttt{final},则关键字 \texttt{if} 可能出现的语言集合为 $\{\texttt{Python}, \texttt{Java}\}$。

设一个单词 $w$ 的可能出现的语言集合为 $S_k$,则代码文件可能的语言即为:

$$A = \bigcup\limits_{w \in \texttt{code}} S_w$$

若 $A$ 为空集,输出 \texttt{Unknown},否则输出 $A$ 中字典序最大的单词即可。

求并集的可以使用 \lstinline{std::set_union} 方便地完成,当然也可以手写模拟。

\section*{标程}

\std{G}

\makesolution
\section*{做法}

假设答案是 $ans$,对于任何 $i \leq ans$ 来说,都有可能正好邀请 $i$ 人,对于任何 $i > ans$ 来说,都不可能正好邀请 $i$ 人。

因为把 $1$ 个人从聚会中移除并不会使任何人不开心。

所以如果我们能邀请 $p_1,p_2,\dots,p_k$,我们也能邀请$p_1,p_2,\dots,p_{k-1}$。

因此可以用二分答案来找到有可能正好邀请到的最大的人数。

假设我们想邀请 $x$ 人,如果有 $i$ 个人比某人 $v$ 穷,就有 $x-1-i$ 人比他富。

$i \leq b_v,x-1-i \leq a_v$ 所以只需要某人 $v$ 满足 $x-1-a_v \leq i \leq b_v$ 即可被邀请。

\section*{标程}

\std{H}

\end{document}