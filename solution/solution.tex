\documentclass{../cpct/ctsol}

\title{ACM 算法实验室 2022 年 12 月月赛题解}
\date{2022 年 12 月 4 日}

\begin{document}

\maketitle
\addsolution{Quento}{AgOH}{dfs}
\addsolution{地铁 \textasciitilde 地铁 \textasciitilde}{Algor}{图的连通性}
\addsolution{数列 \textasciitilde 数列 \textasciitilde}{Algor}{签到}
\addsolution{布丁 \textasciitilde 布丁 \textasciitilde}{Algor}{贪心}
\addsolution{排队 \textasciitilde 排队 \textasciitilde}{Algor}{逆序对}
\addsolution{背包 \textasciitilde 背包 \textasciitilde}{Algor}{构造}
\addsolution{Guesslang}{Zxilly}{模拟}
\addsolution{聚会 \textasciitilde 聚会 \textasciitilde}{Algor}{二分答案}

\section*{题目概览}

\solutiontab

\makesolution
\section*{做法}

对于每个“Quento”棋盘格,从每个数字格为起点做 dfs,统计所有可能出现的结果,对于每个询问查询结果即可。

\section*{标程}

\std{A}

\makesolution
\section*{做法}

\section*{标程}

\std{B}

\makesolution
\section*{做法}

\section*{标程}

\std{C}

\makesolution
\section*{做法}

\section*{标程}

\std{D}

\makesolution
\section*{做法}

\section*{标程}

\std{E}

\makesolution
\section*{做法}

\section*{标程}

\std{F}

\makesolution
\section*{做法}

统计每个可能出现的关键字的可能出现的语言集合,例如若 \texttt{C++} 语言中有关键字 \texttt{if} 和 \texttt{else},\texttt{Java} 语言中有关键字 \texttt{if},\texttt{final},则关键字 \texttt{if} 可能出现的语言集合为 $\{\texttt{Python}, \texttt{Java}\}$。

设一个单词 $w$ 的可能出现的语言集合为 $S_k$,则代码文件可能的语言即为:

$$A = \bigcup\limits_{w \in \texttt{code}} S_w$$

若 $A$ 为空集,输出 \texttt{Unknown},否则输出 $A$ 中字典序最大的单词即可。

求并集的可以使用 \lstinline{std::set_union} 方便地完成,当然也可以手写模拟。

\section*{标程}

\std{G}

\makesolution
\section*{做法}

\section*{标程}

\std{H}

\end{document}
